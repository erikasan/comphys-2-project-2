\documentclass[%
oneside,                 % oneside: electronic viewing, twoside: printing
final,                   % draft: marks overfull hboxes, figures with paths
10pt]{article}

\listfiles               %  print all files needed to compile this document

\usepackage{relsize,makeidx,color,setspace,amsmath,amsfonts,amssymb}
\usepackage[table]{xcolor}
\usepackage{bm,ltablex,microtype}

\usepackage[pdftex]{graphicx}

\usepackage[T1]{fontenc}
%\usepackage[latin1]{inputenc}
\usepackage{ucs}
\usepackage[utf8x]{inputenc}

\usepackage{lmodern}         % Latin Modern fonts derived from Computer Modern

% Hyperlinks in PDF:
\definecolor{linkcolor}{rgb}{0,0,0.4}
\usepackage{hyperref}
\hypersetup{
    breaklinks=true,
    colorlinks=true,
    linkcolor=linkcolor,
    urlcolor=linkcolor,
    citecolor=black,
    filecolor=black,
    %filecolor=blue,
    pdfmenubar=true,
    pdftoolbar=true,
    bookmarksdepth=3   % Uncomment (and tweak) for PDF bookmarks with more levels than the TOC
    }
%\hyperbaseurl{}   % hyperlinks are relative to this root

\setcounter{tocdepth}{2}  % levels in table of contents

% --- fancyhdr package for fancy headers ---
\usepackage{fancyhdr}
\fancyhf{} % sets both header and footer to nothing
\renewcommand{\headrulewidth}{0pt}
\fancyfoot[LE,RO]{\thepage}
% Ensure copyright on titlepage (article style) and chapter pages (book style)

% Ensure copyright on titlepages with \thispagestyle{empty}
\fancypagestyle{empty}{
  \fancyhf{}
  \fancyfoot[C]{{\footnotesize \copyright\ 1999-2021, "Computational Physics I FYS4411/FYS9411":"http://www.uio.no/studier/emner/matnat/fys/FYS4411/index-eng.html". Released under CC Attribution-NonCommercial 4.0 license}}
  \renewcommand{\footrulewidth}{0mm}
  \renewcommand{\headrulewidth}{0mm}
}

\pagestyle{fancy}


% prevent orhpans and widows
\clubpenalty = 10000
\widowpenalty = 10000

% --- end of standard preamble for documents ---


% insert custom LaTeX commands...

\raggedbottom
\makeindex
\usepackage[totoc]{idxlayout}   % for index in the toc
\usepackage[nottoc]{tocbibind}  % for references/bibliography in the toc
\begin{document}

The trial wave function for N particles is given by 
\begin{equation}
 \Psi_T(\mathbf{r})=\Psi_T(\mathbf{r}_1, \mathbf{r}_2, \dots \mathbf{r}_N,\alpha,\beta)
 =\left[
    \prod_i g(\alpha,\beta,\mathbf{r}_i)
 \right]
 \left[
    \prod_{j<k}f(a,|\mathbf{r}_j-\mathbf{r}_k|)
 \right],
 \label{eq:trialwf}
 \end{equation}
where 
\begin{equation}
    g(\alpha,\beta,\mathbf{r}_i)= \exp{[-\alpha(x_i^2+y_i^2+\beta z_i^2)]}=\phi(\mathbf{r}_i)
 \end{equation}
 and 
 \begin{equation}
    f(a,|\mathbf{r}_i-\mathbf{r}_j|)=\Bigg\{
 \begin{array}{ll}
	 0 & {|\mathbf{r}_i-\mathbf{r}_j|} \leq {a}\\
	 (1-\frac{a}{|\mathbf{r}_i-\mathbf{r}_j|}) & {|\mathbf{r}_i-\mathbf{r}_j|} > {a}.
 \end{array}
 \end{equation}
Defining $r_{ij}=|\mathbf{r}_i-\mathbf{r}_j|$ and $u(r_{ij})=\ln{f(r_{ij})}$,
this can be rewritten as 
\begin{equation*}
\Psi_T(\mathbf{r})=\left[
    \prod_i \phi(\mathbf{r}_i)
\right]
\exp{\left(\sum_{j<m}u(r_{jm})\right)}
\end{equation*}
The gradient with respect to particle k is then:
\begin{align*}
  \nabla_k\Psi_T(\mathbf{r})&= \nabla_k\left(\left[\prod_i \phi(\mathbf{r}_i)\right]
\exp{\left(\sum_{j<m}u(r_{jm})\right)}\right)
  \\
  & = \left(\nabla_k\left[\prod_i \phi(\mathbf{r}_i)\right]\right)\exp{\left(\sum_{j<m}u(r_{jm})\right)} + \left[\prod_i \phi(\mathbf{r}_i)\right]\left(\nabla_k
\exp{\left(\sum_{j<m}u(r_{jm})\right)}\right)
  \\
  & = \nabla_k\phi(\mathbf{r}_k)\left[\prod_{i\ne k}\phi(\mathbf{r}_i)\right]\exp{\left(\sum_{j<m}u(r_{jm})\right)}
  \\
  &\qquad
  +  \left[\prod_i\phi(\mathbf{r}_i)\right]
  \exp{\left(\sum_{j<m}u(r_{jm})\right)}\sum_{l\ne k}\nabla_k u(r_{kl})
\end{align*}
where we used the product rule for gradients and the fact that only $\phi(\mathbf{r}_k)$ and $\exp(\sum_{l\ne k}u(r_{kl}))$ are functions of $\mathbf{r}_k$.
\\
The second derivative divided by the trial wave function is then given by 
\begin{align*}
   \frac{1}{\Psi_T(\mathbf{r})}\nabla_k^2\Psi_T(\mathbf{r}) & = \frac{1}{\Psi_T(\mathbf{r})} \nabla_k \cdot \left(\nabla_k\Psi_T(\mathbf{r})\right) \\
   & = \frac{1}{\Psi_T(\mathbf{r})} \nabla_k \cdot \left(\nabla_k\phi(\mathbf{r}_k)\left[\prod_{i\ne k}\phi(\mathbf{r}_i)\right]\exp{\left(\sum_{j<m}u(r_{jm})\right)}\right)
  \\
  &\qquad
  + \frac{1}{\Psi_T(\mathbf{r})} \nabla_k \cdot \left( \left[\prod_i\phi(\mathbf{r}_i)\right]
  \exp{\left(\sum_{j<m}u(r_{jm})\right)}\sum_{l\ne k}\nabla_k u(r_{kl}) \right) \\ 
  & =  \frac{1}{\Psi_T(\mathbf{r})} \left(\nabla_k^2\phi(\mathbf{r}_k)\right)\left[\prod_{i\ne k}\phi(\mathbf{r}_i)\right]\exp{\left(\sum_{j<m}u(r_{jm})\right)}\\ 
  &\qquad +\frac{1}{\Psi_T(\mathbf{r})}\left(\left[\prod_{i\ne k}\phi(\mathbf{r}_i)\right]\exp{\left(\sum_{j<m}u(r_{jm})\right)}\nabla_k\phi(\mathbf{r}_k) \cdot \sum_{l\ne k}\nabla_k u(r_{kl}) \right)\\
  & \qquad+ \frac{1}{\Psi_T(\mathbf{r})}\left(\left[\prod_{i\ne k}\phi(\mathbf{r}_i)\right]\exp{\left(\sum_{j<m}u(r_{jm})\right)}\nabla_k\phi(\mathbf{r}_k) \cdot \sum_{l\ne k}\nabla_k u(r_{kl}) \right)\\
  & \qquad+\frac{1}{\Psi_T(\mathbf{r})} \left[\prod_i\phi(\mathbf{r}_i)\right]
  \exp{\left(\sum_{j<m}u(r_{jm})\right)}\left(\sum_{l\ne k}\nabla_k u(r_{kl})\right)^2 \\
  & \qquad+\frac{1}{\Psi_T(\mathbf{r})} \left[\prod_i\phi(\mathbf{r}_i)\right]
  \exp{\left(\sum_{j<m}u(r_{jm})\right)}\sum_{l\ne k}\nabla_k^2 u(r_{kl})\\
  & = \frac{\nabla_k^2\phi(\mathbf{r}_k)}{\phi(\mathbf{r}_k)}
   + 2\frac{\nabla_k\phi(\mathbf{r}_k)}{\phi(\mathbf{r}_k)}\cdot 
   \sum_{l\ne k}\nabla_k u(r_{kl})
   \\
   &\qquad
   + \sum_{l\ne k}\nabla_k u(r_{kl}) \cdot \sum_{i\ne k}\nabla_k u(r_{ki})
   \\
   &\qquad
   + \sum_{l\ne k}\nabla_k^2 u(r_{kl})
\end{align*}
where we used the product rule for divergences and gradients. The expressions for the gradient of $u(r_{ik})$ can be reexpressed using the chain rule.
We have that, by the chain rule
\begin{align*}
    \frac{\partial u(r_{ki})}{\partial x_k}\mathbf{\hat{x}_k}=\frac{\partial r_{ki}}{\partial x_k}\frac{\partial u(r_{ki})}{\partial r_{ki}}\mathbf{\hat{x}_k}=u'(r_{ki})\frac{x_k-x_i}{r_{ki}}\mathbf{\hat{x}_k}
\end{align*}
and we get the equivalent expressions for the partial derivatives with respect to $y_k$ and $z_k$ due to symmetry. Combining these, we get 
\begin{align*}
    \nabla_k u(r_{ki})=\frac{(\mathbf{r}_k-\mathbf{r}_i)}{r_{ki}}u'(r_{ki})
\end{align*}
Similarly, we get for the second partial derivative that 
\begin{align*}
    \frac{\partial^2 u(r_{ki})}{\partial x_k^2}&=\frac{\partial }{\partial x_k}\left(u'(r_{ki})\frac{x_k-x_i}{r_{ki}}\right)= \frac{\partial u'(r_{ki})}{\partial x_k}\frac{x_k-x_i}{r_{ki}}+u'(r_{ki})\frac{\partial }{\partial x_k}\left(\frac{x_k-x_i}{r_{ki}}\right)\\ &= u''(r_{ki})\frac{(x_k-x_i)^2}{r_{ki}^2}+u'(r_{ki})\left(\frac{1}{r_{ki}}-\frac{(x_k-x_i)^2}{r_{ki}^3}\right)
\end{align*}
and again, we get equivalent expressions for the second partial derivatives with respect to $y_k$ and $z_k$. Because $(x_k-x_i)^2+(y_k-y_i)^2+(z_k-z_i)^2=r_{k_i}^2$, the Laplacian becomes
\begin{equation*}
    \nabla_k^2 u(r_{kl})=u''(r_{ki})\frac{r_{ki}^2}{r_{ki}^2}+u'(r_{ki})\left(\frac{3}{r_{ki}}-\frac{r_{ki}^2}{r_{ki}^3}\right)= u''(r_{kj})+\frac{2}{r_{kj}}u'(r_{kj})
\end{equation*}
Inserting this in the expression for the second derivative, we get Morten's result.\\
Here are the analytic expressions for the quantum force and the second derivative:
The quantum force $\frac{2\nabla\psi}{\psi}$ for particle $k$ is given by:
\begin{equation*}
\begin{split}
 \frac{2\nabla_k\psi}{\psi} &=\frac{2\nabla_k\phi(\mathbf{r}_k)\left[\prod_{i\ne k}\phi(\mathbf{r}_i)\right]\exp{\left(\sum_{j<m}u(r_{jm})\right)} +  2\left[\prod_i\phi(\mathbf{r}_i)\right]
  \exp{\left(\sum_{j<m}u(r_{jm})\right)}\sum_{l\ne k}\nabla_k u(r_{kl})}{\psi}\\
  &= \frac{-4\alpha(x_k \hat{x}_k+y_k \hat{y}_k+\beta z_k \hat{z}_k)\psi+2\psi\sum_{l\ne k}\nabla_k u(r_{kl})}{\psi}\\
  &=-4\alpha(x_k \hat{x}_k+y_k \hat{y}_k+\beta z_k \hat{z}_k)+2\sum_{l\ne k}\frac{(\mathbf{r}_k-\mathbf{r}_l)}{r_{kl}}u'(r_{kl})\\
  &= -4\alpha(x_k \hat{x}_k+y_k \hat{y}_k+\beta z_k \hat{z}_k)+2\sum_{l\ne k}\frac{(\mathbf{r}_k-\mathbf{r}_l)}{r_{kl}}\frac{-a}{ar_{kl}-r^2_{kl}}\\
  &= -4\alpha(x_k \hat{x}_k+y_k \hat{y}_k+\beta z_k \hat{z}_k)-\sum_{l\ne k}\frac{2a}{ar^2_{kl}-r^3_{kl}}{(\mathbf{r}_k-\mathbf{r}_l)}
\end{split}
  \end{equation*}
Now let's have a good look at the best that we call second derivative / kinetic energy. 
We have the formula given by
\begin{align*}
\frac{1}{\Psi_T(\mathbf{r})}\nabla_k^2\Psi_T(\mathbf{r}) &=
   \frac{\nabla_k^2\phi(\mathbf{r}_k)}{\phi(\mathbf{r}_k)}
   + 2\frac{\nabla_k\phi(\mathbf{r}_k)}{\phi(\mathbf{r}_k)}\cdot 
   \sum_{l\ne k}\nabla_k u(r_{kl})
   \\
   &\qquad
   + \sum_{l\ne k}\nabla_k u(r_{kl}) \cdot \sum_{i\ne k}\nabla_k u(r_{ki})
   \\
   &\qquad
   + \sum_{l\ne k}\nabla_k^2 u(r_{kl})
   \\
   &=\left[ 4\alpha^2\left[x_k^2+y_k^2+\beta^2z_k^2\right]-4\alpha-2\alpha\beta\right]-4\alpha(x_k \hat{x}_k+y_k \hat{y}_k+\beta z_k \hat{z}_k)\cdot \sum_{j\ne k}\frac{-a}{ar^2_{kj}-r^3_{kj}}{(\mathbf{r}_k-\mathbf{r}_j)}
   \\
   &\qquad + \sum_{k \ne i}\sum_{k \ne j}{(\mathbf{r}_k-\mathbf{r}_i)\cdot(\mathbf{r}_k-\mathbf{r}_j)}\frac{a^2}{(ar^2_{kj}-r^3_{kj})(ar^2_{ki}-r^3_{ki})}
   \\
   &\qquad + \sum_{k \ne j}\frac{a(a-2r_{kj})}{r_{kj}^2(a-r_{kj})^2}-\frac{2a}{(ar^2_{kj}-r^3_{kj})}\\
   &=\left[ 4\alpha^2\left[x_k^2+y_k^2+\beta^2z_k^2\right]-4\alpha-2\alpha\beta\right]-4\alpha(x_k \hat{x}_k+y_k \hat{y}_k+\beta z_k \hat{z}_k)\cdot \sum_{j\ne k}\frac{-a}{ar^2_{kj}-r^3_{kj}}{(\mathbf{r}_k-\mathbf{r}_j)}
   \\
   &\qquad + \sum_{k \ne i}\frac{-a(\mathbf{r}_k-\mathbf{r}_i)}{(ar^2_{ki}-r^3_{ki})}\cdot\sum_{k \ne j}\frac{-a{(\mathbf{r}_k-\mathbf{r}_j)}}{(ar^2_{kj}-r^3_{kj})}
   \\
   &\qquad + \sum_{k \ne j}\frac{-a^2}{r_{kj}^2(a-r_{kj})^2}
\end{align*}
 \end{document}
